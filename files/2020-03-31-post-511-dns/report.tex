\documentclass[12pt,oneside,a4paper,english]{article}
\usepackage[T1]{fontenc}
\usepackage[latin2]{inputenc}
\usepackage[margin=2.25cm,headheight=26pt,includeheadfoot]{geometry}
\usepackage[english]{babel}
\usepackage{listings}
\usepackage{color}
\usepackage{titlesec}
\usepackage{titling}
\usepackage[framed, numbered]{matlab-prettifier}
\usepackage{changepage}
\usepackage{amsmath}
\usepackage{hyperref}
\usepackage{enumitem}
\usepackage{graphicx}
\graphicspath{{figures/}}
\usepackage{fancyhdr}
\usepackage{lastpage}
\usepackage{caption}
\usepackage{tocloft}
\usepackage{setspace}
\usepackage{multirow}
\usepackage{titling}
\usepackage{float}
\usepackage{comment}
\usepackage{booktabs}
\usepackage{indentfirst}
\usepackage{lscape}
\usepackage{booktabs,caption}
\usepackage[flushleft]{threeparttable}
\usepackage[english]{nomencl}
\usepackage{xcolor}
\usepackage{lipsum}
\usepackage{placeins}
\usepackage{setspace}
\usepackage{ar}
\DeclareMathOperator{\sign}{sign}

% --- set footer and header ---
\pagestyle{fancy}
\fancyhf{}

\setlength{\parindent}{2em}
\title{Direct Numerical Simulaiton} % to reference as \title, dont use \maketitle
\makeatletter\let\Title\@title\makeatother



\lstset{language=Matlab,
style=Matlab-editor,
basicstyle=\normalsize\mlttfamily,
numbers=left,
numberstyle={\scriptsize\color{black}},			% size of the numbers
numbersep=0.5cm
}

\newlist{steps}{enumerate}{1}
\setlist[steps, 1]{leftmargin=1.5cm,label = Step \arabic*:}
\renewcommand{\headrulewidth}{1pt}
\renewcommand{\footrulewidth}{1pt}

%\lhead{\Title}
\lhead{Mohammad Zandsalimy}
\rhead{\Title}
\rfoot{\includegraphics[height=1.25cm]{root/logo.pdf}} % right header logo
\setlength\headheight{16pt}
\setlength{\footskip}{50pt}
\cfoot{\thepage}

% --- End of page settings ---



\begin{document}
\pagenumbering{roman}
\input{sources/0_frontpage.tex}

\newpage
\addcontentsline{toc}{section}{Abstract}
\section*{Abstract} \label{abstract}
Direct numerical simulation (DNS) of turbulent flows is reviewed in the present study. DNS is a simulation in computational fluid dynamics in which the Navier-Stokes equations are numerically solved without any turbulence modeling. The full range of spatial scales in turbulent flows should be resolved numerically. There are also several issues arising from boundary condition implementation as well as time integration. All these properties make DNS a very computationally intensive numerical method. DNS has enabled the scientific community to get a deeper insight into the more complicated flow physics which are not easily attained using conventional numerical or even experimental methods. A few examples are discussed and the results are presented. Furthermore, DNS has largely impacted turbulence models and provided further insight into the structure of turbulent flows. Although a great tool for analyzing different fluid flows, at the moment, DNS may not provide a suitable solution to real-world problems, mainly due to high numerical solution costs.

Keywords: Direct Numerical Simulation, Turbulence Modeling, Computational Fluid Dynamics, Navier Stokes Equations
\thispagestyle{fancy}

\newpage
\doublespacing
%\addcontentsline{toc}{section}{Table of Contents}
\renewcommand{\baselinestretch}{1}\normalsize
\tableofcontents
\renewcommand{\baselinestretch}{1}\normalsize
%\singlespacing
\thispagestyle{fancy} % force page style

\newpage
\addcontentsline{toc}{section}{List of Figures}
\listoffigures
\thispagestyle{fancy}

\newpage
\addcontentsline{toc}{section}{List of Tables}
\listoftables
\thispagestyle{fancy}

\newpage
\addcontentsline{toc}{section}{Nomenclature}
\input{sources/9_nomenclat.tex}
\thispagestyle{fancy}








\clearpage
\pagenumbering{arabic}

\section{Introduction}
A direct numerical simulation (DNS) is a simulation in computational fluid dynamics in which the Navier-Stokes equations are numerically solved without any turbulence modeling. This means that the whole range of spatial and temporal scales of the turbulence must be resolved through the numerical solution. This includes spatial scales as small as Kolmogorov scales, and as large as the integral scale, associated with the motions containing most of the kinetic energy. It is generally accepted that DNS results play an important role in the enhancement and calibration of statistical and subgrid-scale turbulence models needed to provide closure of Reynolds-averaged and filtered transport equations solved in RANS, Detached Eddy Simulation (DES) and Large-Eddy Simulation (LES). DNS results also play an important role in identifying resolution requirements for thermal and kinetic turbulent boundary layers \cite{wagner2011}. In the present study, we will try to shine an oblique light on DNS and uncover a few of its properties as a full review of the method will take much more time and effort than what we have here.

At the end of the last millennium, it was shown that predictions by direct numerical simulation agree well with experimental results obtained with laser doppler anemometry and particle image velocimetry if low Reynolds number flows are considered \cite{eggels_unger_weiss_westerweel_adrian_friedrich_nieuwstadt_1994}. Despite the widely accepted merit of DNS for fundamental flow studies, the technique has not yet found its way into industrial flow problems (mainly with high Reynolds numbers). The reason for this is the required computational resource increment with approximately the third power of the Reynolds number. For example, any problem concerning aircraft or vehicle aerodynamics is characterized by very high Reynolds numbers. In this regard, Spalart \cite{SPALART2000252} estimated in the year 1999, that it will take until 2080 for DNS to apply to ``real-world'' flows. However, in the last years, several DNS studies have been performed which are relevant for various industrial branches. The common objective of these flow simulations was to produce a reliable and comprehensive flow database for the validation and improvement of corresponding turbulence models. These models are known to perform well for simple shear flows but do not give good results in general cases.

The earliest traces of DNS are coming from the US National Center for Atmospheric Research \cite{Kardashian1998}. Orszag and Patterson \cite{PhysRevLett76} in 1972 performed a simulation of isotropic turbulence at a Reynolds number of 35. The resolution in use is inadequate by today's standards, but the calculations demonstrated how spectral methods could be used to perform largescale computations of three-dimensional turbulence. The computing resources of that era did not allow DNS of wall-bounded turbulence, however, coarse-grid computations of free-shear layers could be performed \cite{RILEY1980}. One of the most important scientists in this field is Spalart who developed a method to compute the turbulent flat-plate boundary layer under zero and favorable pressure gradients \cite{spalart_1988}. His numerical results have been widely used since with almost 2000 full-text reviews of that single article. All of these simulations were homogeneous in the streamwise direction with periodic boundary conditions. Computing flows that are inhomogeneous in the streamwise direction require turbulence to be specified at the inflow plane \cite{Kardashian1998}. This results in the simulation of reasonably complex flows such as flow over a backward-facing step \cite{le_moin_kim_1997}, and flat plate boundary layer separation \cite{kitsios_sekimoto_atkinson_2017}.

However, there are several DNS projects at high Reynolds numbers, including the attachment line of swept wings by Spalart in 1989 \cite{spalart1989direct}. DNS was applied at the (local) Reynolds number of the flow on an airliner. This is a case of ``microscopic'' simulation, in which it is justified to isolate a very small region of the flow. Spalart once received dubious praise for simulating ``a milli-second over a postage stamp'' \cite{laurence1999engineering}. Simulations of homogeneous turbulence and boundary layers could also be described as microscopic. DNS of a full aerodynamics body is usually out of the question (unless in the case of a simple sphere or cylinder). Recently, DNS has been utilized for turbulence simulation using the Lattice Boltzmann method \cite{wang2014direct} in wall bounded flows at Re=180. They showed the ability and validity of LBM on simulating turbulent flows.







\section{Overview}
DNS is a simulation with sufficient spatial and temporal resolution to capture the smallest scales of the turbulent flow. Fortunately, the smallest scales in any flow are finite and capturable. There is a good correlation between experiments and DNS. However, this does not mean the same temporal and spatial development of the dependent variables in the computational domain for DNS compared to experiments. The solutions of the nonlinear equations of fluid motion are chaotic which implies that the agreement between DNS and experiment is related to the statistical behavior of the solution and measured data. Just like any other chaotic system, a small change in initial or boundary conditions, numerical method, and numerical grid, can result in a completely different solution. However, as long as the accuracy of the numerical simulation is not compromised, the statistical properties remain unchanged. Of all the methods for simulating turbulence, it is the easiest approach conceptually, since there is no averaging of the equations and therefore no closure modeling.

\noindent
Some advantages of direct numerical simulation are presented as,
\begin{itemize}
\item In theory, all the values of each dependent variable are accessible at each point in space and time. This allows the detailed exploration of the physics and flow structures.
\item Any statistic of interest can be computed. Although the accuracy and statistical convergence can be an issue.
\item Flow parameters can be easily varied (within a limited range of course). For example, the Reynolds number can be easily varied but is limited by the numerical resolution.
\item Experimental conditions are perfectly controllable.
\item When compared to other computational approaches, the closure problem is avoided.
\end{itemize}


\noindent
However, there are major disadvantages with direct numerical simulation,
\begin{itemize}
\item The spatial and temporal resolution limits the Reynolds number (or the other way round if you want).
\item Difficulty or impossible implementation of direct numerical simulations in complex geometries. The need to use very accurate numerical methods are often incompatible with their implementation in complex geometries.
\end{itemize}








\section{Andrey Kolmogorov}
Andrey Nikolaevich Kolmogorov was a Soviet mathematician who made significant contributions to the mathematics of probability theory, topology, intuitionistic logic, turbulence, classical mechanics, algorithmic information theory and computational complexity. His publications significantly influenced the field of turbulence in fluid dynamics. In classical mechanics, he is best known for the Kolmogorov-Arnold-Moser (KAM) theorem, first presented in 1954 at the International Congress of Mathematicians \cite{Chierchia2008}. KAM theorem is a result in dynamical systems about the persistence of quasiperiodic motions under small perturbations. The theorem partly resolves the small-divisor problem that arises in the perturbation theory of classical mechanics. The problem is whether or not a small perturbation of a conservative dynamical system results in a lasting quasiperiodic orbit. The original breakthrough to this problem was given by Andrey Kolmogorov in 1954  \cite{Chierchia2008}. This was later continued by Vladimir Arnold (Kolmogorov's student) in 1963 (for analytic Hamiltonian systems) \cite{ASNSP_1994_4_21_4_541_0}. Figure \ref{fig_kol_hiking_1} is photo of Kolmogorov hiking in the Caucasus area situated between the Black Sea and the Caspian Sea.

\begin{figure}[ht]
    \centering
    \includegraphics[width = 0.45\textwidth]{kolmogorov.jpeg}
    \caption[Kolmogorov hiking in the Caucasus area ]{Kolmogorov hiking in the Caucasus area situated between the Black Sea and the Caspian Sea \cite{Kendall1991}.}
    \label{fig_kol_hiking_1}
\end{figure}

The name Kolmogorov has different meanings in different mathematical and scientific communities. For those interested in the turbulence and fluid flow, Kolmogorov will always be remembered for the theory of universal equilibrium of the small-scale components of fluid motion that he put forward in 1941. This powerful theory, which has been found extremely useful in a wide variety of physical contexts, was published (in English) in two short notes \cite{Kendall1991}. These papers are unusual for a mathematician because they contain no mathematics to speak of, dimensional analysis and elementary probability concepts being all that is in there. They are essentially a statement of two hypotheses, justified heuristically. Kolmogorov published four other similar short notes on turbulence, the last being a reconsideration and refinement of the universal equilibrium theory \cite{Kendall1991}. All these papers have had by far the greatest impact on the study of turbulence phenomena.

The basis of Kolmogorov's two hypotheses was the notion of kinetic energy cascading from components of the large length scale structures in the turbulent flow to components with smaller length scales as a consequence of nonlinear inertial interaction of these components. Kolmogorov recognized that increasing Reynolds number of the flow results in a decrement in the smallest length scale present in the flow, which in turn increases the number of steps in the energy cascade. Kolmogorov concluded that the small length scales of turbulent flow are asymptotically homogeneous in space and time and isotropic, regardless of the way the turbulence is being generated and regardless of its large-scale statistical properties \cite{pope_2000}. This extremely powerful premise represents the essence of the Kolmogorov theory. From a practical point of view, it has the weakness of referring only to components on small length scales, which normally make negligible contributions to the rates of transfer of momentum and mass in fluid flow fields. However, they describe the small length scales perfectly and are essential in turbulence studies.






\section{Kolmogorov Microscales}
At the Kolmogorov scale, viscosity is dominant and kinetic energy dissipates in the form of heat. These scales are defined as in table \ref{table_kol_scale_1}. where $\varepsilon$ is the average rate of dissipation of kinetic energy per unit mass, and $\nu$ is the kinematic viscosity.
Andrey Kolmogorov introduced the idea that the smallest scales of turbulence are universal and that they depend on $\varepsilon$ and $\nu$ only. The definitions of the Kolmogorov microscales can be obtained using this idea and dimensional analysis. Kinematic viscosity has a dimension of $\dfrac{\text{length}^2}{\text{time}}$, and the dimension of the energy dissipation rate per unit mass is  $\dfrac{\text{length}^2}{\text{time}^3}$. The only combination that has the dimension of time is $\tau_{\eta} = \left(\dfrac{\nu}{\varepsilon}\right)^{\frac{1}{2}}$ which is the Kolmogorov time scale. Similarly, the Kolmogorov length scale is the only combination of $\varepsilon$ and $\nu$ that has a dimension of length. An example of the Kolmogorov length scale for atmospheric fluid flow (in which the large eddies have length scales in the kilometer range) range from 0.1 to 10 millimeters. For flows in a smaller size, however, such as in laboratory systems, length scales can be much smaller \cite{Gibson1929}.

\begin{table}[ht]
    \centering
    \caption{Different Kolmogorov scales.}
    \label{table_kol_scale_1}
    \begin{tabular}{lc}
    \hline
    Kolmogorov scale & Relation \\
    \hline  \hline \\[-10pt]
     Length scale & $\eta = \left(\dfrac{\nu^3}{\varepsilon}\right)^{\frac{1}{4}}$ \\[10pt]
     Time scale & $\tau_{\eta} = \left(\dfrac{\nu}{\varepsilon}\right)^{\frac{1}{2}}$  \\[10pt]
     Velocity scale & $u_{\eta} = \left(\nu \varepsilon\right)^{\frac{1}{4}}$ \\[10pt]
    \hline \\
\end{tabular}
\end{table}

Alternatively, the definition of the Kolmogorov time scale can be obtained from the inverse of the mean square strain rate tensor as $\tau _{\eta}=(2\langle E_{ij}, E_{ij}\rangle)^{\frac{-1}{2}}$. Using the definition of the energy dissipation rate per unit mass ($\varepsilon =2\nu \langle E_{ij}, E_{ij}\rangle$), the previous relation gives $\tau _{\eta}=(\dfrac{\nu}{\varepsilon})^{\frac{1}{2}}$. Further, the Kolmogorov length scale can be obtained as the scale at which the Reynolds number is equal to 1, $\text{Re}=\dfrac{UL}{\nu}=\dfrac{\eta }{\tau _{\eta }}\dfrac{\eta }{\nu} =1$.







\section{Spatial Resolution}
In DNS we need to be very careful with the spatial resolution of the numerical mesh. The range of scales that need to be accurately represented in a computation comes from physics and boundary conditions. The numerical grid should be able to capture every single phenomenon happening in the fluid domain. The Kolmogorov length scale is commonly used as the smallest scale that needs to be resolved in the turbulent flow. Does this mean that the grid should be at least smaller than the smallest length scales present in the turbulent flow? Not really! Table \ref{table_first_11} presents the resolution used in a few wall-bounded problems which are solved using spectral DNS. The resolution in these simulations is larger but of the same order as Kolmogorov scales. These simulations do not capture the exact Kolmogorov scales but still give very good results. It is mentioned that to get a good result out of DNS, ``most'' of the dissipation should be captured (not all of it). As stated by \cite{Kardashian1998} most of the dissipation in the curved channel occurs at scales greater than $15\eta$ (calculated based on average dissipation). Furthermore, physical parameters such as channel width, boundary layer thickness, or mixing layer thickness determine the largest scales.

\begin{table}[ht]
    \centering
    \caption[The resolution used in spectral DNS of some homogeneous and wall-bounded flows]{The resolution used in spectral DNS of some homogeneous and wall-bounded flows ($\eta$ is calculated at the wall) \cite{Kardashian1998}.}
    \label{table_first_11}
    \begin{tabular}{ll}
    \hline
    Flow & Resolution ($\times \eta$) \\
    \hline \hline \\
    Curved channel & $\Delta z = 3.75$, $\Delta r = 0.13$, $r_c \Delta \theta = 11.25$ \\
    Plane channel & $\Delta x = 7.5$, $\Delta y = 0.03$, $\Delta z = 4.4$ \\
    Boundary layer & $\Delta x = 14.3$, $\Delta y = 0.33$, $\Delta z = 4.8$ \\
    Homogeneous shear & $\Delta x = 7.8$, $\Delta y = 3.9$, $\Delta z = 3.9$ \\
    Isotropic turbulence & $\Delta x = 4.5$, $\Delta y = 4.5$, $\Delta z = 4.5$ \\
    \hline \\
\end{tabular}
\end{table}

All of the simulations in table \ref{table_first_11} have utilized spectral DNS, which gives high convergence rate. Differencing schemes with larger numerical errors require higher resolution to achieve the same level of accuracy. We should take care of capturing the smallest wavenumber in the flow field for an accurate simulation. Assume for example, that a wave with wavelength $3\eta$ is required to be captured to 5\% accuracy. It can be shown that the second-order central difference scheme requires a mesh spacing equal to $0.26\eta$ to meet this requirement, while the fourth-order central difference, sixth-order Pad'e, and Fourier spectral schemes require mesh spacings of $0.55\eta$, $0.95\eta$, and $1.5\eta$, respectively \cite{Kardashian1998}.

The error at small scales includes parts other than differentiation error that result from nonlinearity. DNS requires that the mesh be fine enough for this error to be small. Another important source of error is aliasing. Nonlinear operations generate modes that are not in the set of modes being represented. This happens when functions are expressed in terms of a finite number of basis functions. The small scales have larger levels of aliasing error. Finite difference schemes typically have lower levels of aliasing error than spectral methods \cite{Kardashian1998}.

Suppose a flow domain is required to be simulated using DNS. The integral scale of the flow is $L$. If we have $N$ cells in the domain, $Nh>L$ must hold so that the integral scale is contained within the computational domain. On the other hand, $h \leq \eta$ must hold so that the Kolmogorov scales are captured. We know that the rate of kinetic energy dissipation is defined by $\varepsilon = \dfrac{u'^3}{L}$ in which $u'$ is the root mean square of the flow velocity. As a result, the number of cells in three dimensions must satisfy the following relation.

\begin{equation}
N^3 \geq \text{Re}^{\frac{9}{4}}
\end{equation}

In which Reynolds number is defined by the RMS velocity as $\text{Re}=\dfrac{u' L}{\nu}$. This shows the high memory and computational cost of DNS calculations even at low Reynolds numbers. For the Reynolds numbers encountered in most industrial applications, the computational resources required by a DNS would exceed the capacity of the most powerful computers currently available.







\section{Time Integration}
Turbulent flow simulation includes many different time scales. This property makes turbulence simulation a stiff problem. Stiff problems in CFD are mainly tackled with implicit time advance schemes \cite{BAKER416}. It is only logical if we do a similar approach in DNS. A good property in implicit time advancing is the ability to use large time steps. Unfortunately, the requirement of time accuracy over a wide range of time scales does not allow very large timesteps in DNS. Large timesteps can result in large errors in the small time scales.

Von Neumann stability analysis is used to illustrate the effect of time advancement on the error at different scales. Consider the one-dimensional advection equation on a periodic setting. The solution to this problem can be represented as a summation of Fourier modes. The amplification factor can be obtained is a function of $kh$ and the CFL number ($c\dfrac{\Delta t}{h}$). In these relations, $k$ is the modified wave number, $h$ is the mesh size, and $c$ is the speed of sound. The amplitude and phase errors introduced by the time-discretization method can be examined as a function of $kh$ for different values of the CFL number. This idea is depicted in figure \ref{fig_amplitude_four_1} where error amplitude is presented when a combination of the sixth-order Pad'e scheme and fourth-order Runge-Kutta scheme is used for spatial differencing and time integration, respectively. This figure shows the strong influence of timestep on small scale errors \cite{Kardashian1998}. This figure depicts the amplification factor vs. $kh$. Note that the amplification factor of the exact solution is 1.

\begin{figure}[H]
    \centering
    \includegraphics[width = 0.65\textwidth]{amp_kh.eps}
    \caption[The influence of implicit time integration on small scale accuracy]{The influence of implicit time integration on small scale accuracy \cite{Kardashian1998}.}
    \label{fig_amplitude_four_1}
\end{figure}

However, this does not imply that explicit time advance schemes should be utilized for turbulence simulation. A common method in incompressible DNS of wall-bounded flows is to use implicit time advancement for the diffusion terms and explicit time advancement for the convection terms \cite{TWORZYDLO1992245}. A general statement about the appropriateness of using implicit time advancement for the convection terms cannot be made. As presented by \cite{choi1994effects} very large timesteps were found to cause the turbulence in the channel to decay to a laminar state. They also showed that using explicit time integration might not work for specific flow conditions. In such problems, a fully implicit scheme should be adopted.

\section{Boundary conditions}
One of the most challenging issues in DNS is the application of specific boundary conditions. Among boundary condition types, periodic boundaries are easy to implement in DNS. Accurate and efficient ways of treating the far-field boundaries in fluid domains such as boundary layers, mixing layers, and wakes, exist. The main difficulty is posed by the inflow and outflow boundary conditions. The only correct inflow boundary condition in a turbulent domain is the exact solution \cite{jiang1999non} which is unknown. If complex flows are required to be computed, turbulent inflow and outflow boundary conditions are necessary. To mitigate this problem, early DNS methods assumed that turbulence is homogeneous in the streamwise direction which gives a temporal simulation. Then, a convection velocity was used to relate the temporal to the spatial evolution to match experiments. This approach works very well for a limited class of flows such as decaying grid turbulence, homogeneous shear flows, plane mixing layers, and turbulence passing through axisymmetric contractions and expansions \cite{Kardashian1998}. This approach, however, does not simulate the non-homogenous flow in the streamwise direction in experiments. Spalart \cite{spalart_1988} presented an ingenious coordinate transformation to overcome this limitation in DNS of the turbulent boundary layer. The transformation allowed periodic boundary conditions in the streamwise direction.

The compressible flow will introduce other issues in boundary conditions. The characteristic analysis can be used to determine boundary conditions in such flows. In a compressible flow field, the acoustic waves influence the solution at the inflow and outflow boundaries. Reflection of acoustic waves back into the flow domain can have dramatic outcomes. The Taylor hypothesis runs into problems in a compressible setting. This is because acoustic waves and physical vortices travel at different speeds. According to literature, using a single convection velocity for the inflow boundary is only valid for incompressible case (or low compressibility). Much work has been done in developing non-reflecting boundaries for inflow and outflow conditions in DNS \cite{Kardashian1998}.

\section{DNS Results}
In this section, a few DNS experiments are presented and the results are discussed. DNS has been the main focus of some research groups for a long time. The biggest issue right now is the computational power of today's computers. In the coming years, with increasing computational efficiency and power of new computers, we will definitely see more and more DNS studies.

\subsection{Droplet deformation and evaporation in isotropic turbulence}
Albernaz et al. used a hybrid LBM to study the deformation and evaporation of a single droplet in stationary isotropic turbulence. In their hybrid method, the fluid density and velocity fields were obtained via the lattice Boltzmann method (LBM) with a D3Q19 lattice. This is a 3-dimensional configuration of  LBM with 19 vectors of interaction between points inside the domain. In other words, a fluid ``particle'' has 18 possible velocity directions, plus a zero velocity. The internal energy equation contained a correction term proportional to the difference between the mean pressure of the domain and the initial reference pressure. The correction is needed for conditions close to the critical point where fluctuations of thermodynamic properties occur. A statistically stationary velocity field for $73<\text{Re}<133$ was utilized in the simulations. The pseudopotential method was used to simulate the droplet in the LBM methodology. The liquid droplet was surrounded by its vapor as the carrier fluid. The interface between the liquid and vapor was considered as a thin transition layer of finite width where the density changes smoothly from one phase to the other. The ratio of the liquid to vapor density was about 10. This was true for the dynamic viscosity ratios as well. The surface tension was calculated via the Young-Laplace equation, which relates the pressure jump across the interface to the product of surface tension and the local curvature. And finally, the initial droplet diameter ranged from $50\eta$ to $80\eta$ \cite{albernaz2017droplet}.

This study has a few interesting results. For a fixed Re, increasing the initial diameter of the droplet ($d_{\circ}$) increases the kinetic energy of the carrier fluid and reduces the kinetic energy of the droplet. Droplet deformation increases with increasing $d_{\circ}$ due to the increase in Weber's number. Reducing the surface tension increases the fluctuations of the thermodynamic properties, thus increasing the evaporation rate. At the droplet surface, low-temperature regions are associated with stronger curvature, whereas higher temperature occurs in flatter surface regions. Droplet volume fluctuations are correlated with vapor temperature fluctuations. Strong correlations occur between positive temperature fluctuations and vapor condensation. Figure \ref{fig_Albernaz_2017_1} shows the temperature distribution over the droplet surface for different values of the deformation parameter, $S^* = \dfrac{S - S_{\circ}}{S_{\circ}}$, where S is the instantaneous area of the droplet surface and $S_{\circ}$ is the equivalent surface area of a sphere whose volume is identical to that of the deformed droplet.

\begin{figure}[H]
    \centering
    \includegraphics[width = 0.95\textwidth]{Albernaz_2017.jpg}
    \caption[Temperature distribution over the droplet surface]{Temperature distribution over the droplet surface for different values of the deformation parameter \cite{albernaz2017droplet}.}
    \label{fig_Albernaz_2017_1}
\end{figure}




\subsection{Rayleigh-Taylor mixing}
Youngs used two numerical approaches to simulate Taylor-mixing, including direct numerical simulation and implicit large eddy simulation. They showed the results for four test cases, single-mode Rayleigh-Taylor instability, self-similar Rayleigh-Taylor mixing, three-layer mixing, and a tilted-rig Rayleigh-Taylor experiment. It was found that both approaches give similar results for the high-Reynolds number behavior. For DNS the Navier-Stokes equations are used and include the effects of viscosity, species diffusion, and heat conduction. Simplified material properties are used. The same diffusivity is used for all fluid species and for heat. This leads to a simplified form of the equations \cite{youngs2017rayleigh}.

One of their interesting simulations was three-fluid mixing. Three fluids are initialized in the 3-dimensional domain with three different densities. Two fluid with densities $\rho_1$ and $\rho_2$ occupy a quarter of the height of the domain, each, while the third fluid (with density $\rho_3$) occupies half of the domain. A range of density ratios was considered. Calculations are run to $\tau = 10$, when the amount of mixing is close to the final value. Figure \ref{fig_Youngs_1} shows volume fraction distributions for DNS on a $2048 \times 1024 \times 1024$ mesh and Re$=3400$. In this test, $\rho_1=\rho_3=\dfrac{1}{3} \rho_2$.

\begin{figure}[H]
    \centering
    \includegraphics[width = 0.95\textwidth]{Youngs.jpeg}
    \caption[Dense fluid volume fraction (plane sections) for DNS]{Dense fluid volume fraction (plane sections) for DNS \cite{youngs2017rayleigh}.}
    \label{fig_Youngs_1}
\end{figure}






\subsection{Flow around an airfoil}
Hosseini et al. conducted a three-dimensional direct numerical simulation to study the turbulent flow around the NACA4412 wing section at a moderate Reynolds number of Re$=400,000$ at 5 degrees angle of attack. The mesh was optimized to properly resolve all relevant scales in the flow and included about 3.2 billion grid points. The incompressible spectral-element Navier-Stokes solver was used to conduct the simulation. An unsteady volume force is used to trip the flow to turbulence on both sides of the wing at a 10\% chord location. The Reynolds numbers on the suction side is about 373 and about 346 on the pressure side. The effect of adverse pressure gradients on the mean flow was studied. They essentially showed the potential of high-order (spectral) methods in simulating complex external flows at moderate Reynolds numbers \cite{hosseini2016direct}.

Figure \ref{fig_ussein_bolt_1} shows an instantaneous visualization of vortical structures. The smoothness of the hairpin vortices, characteristic of transitional flows, and of the smallest structures in the turbulent region shows that the mesh is appropriate to capture all the relevant flow features. It is interesting to observe how the turbulent vortical structures occupy a significantly larger volume on the suction side than on the pressure side, due to the effect of adverse pressure gradient on the boundary layer thickness. This also results in more powerful and energetic structures outside the boundary layer. A closeup view of the wake region of the airfoil is shown in figure \ref{fig_ussein_bolt_2}. Near the trailing edge the structures on the upper side travel through a low-speed region, colored in blue. The near-wall structures from both sides start to interact as they leave the trailing edge, and a region of reverse flow right after the trailing edge of the wing is formed. Further downstream a von Karman vortex street is formed.

\begin{figure}[H]
    \centering
    \includegraphics[width = 0.95\textwidth]{ussein_bolt_1.jpeg}
    \caption[Instantaneous vortical structures]{Instantaneous vortical structures, colored with chordwise velocity, ranging from $-0.1$ (dark blue) to $1.5$ (red) \cite{hosseini2016direct}.}
    \label{fig_ussein_bolt_1}
\end{figure}

\begin{figure}[H]
    \centering
    \includegraphics[width = 0.95\textwidth]{ussein_bolt_2.jpeg}
    \caption[Instantaneous vortical structures]{Instantaneous vortical structures, colored with chordwise velocity, ranging from $-0.1$ (dark blue) to $1.1$ (red) \cite{hosseini2016direct}.}
    \label{fig_ussein_bolt_2}
\end{figure}









\section{Conclusion}
There have been several studies on direct numerical simulation in different areas of research including turbulence modeling. An important thing to note is that researchers from many different disciplines are looking into DNS with articles on different branches of science. Utilizing DNS enables us to readily test new ideas and theories on the flow field and we can be confident about our findings since there is no modeling of flow structures. There is a bright future for DNS, definitely. The ever-growing efficiency and power of computers make DNS more feasible as we speak. Aeroacoustics, flow control, high-speed flows, reacting flows, bio-fluidics, and many other applications can benefit a lot from DNS. The numerical methods used in DNS are evolving as the flow geometries become more complex. This comes mainly from the significantly higher numerical fidelity required by DNS. In the present study, a few examples of different DNS applications were introduced. However, there are many other interesting applications of this numerical method that require more time and effort to review.









\newpage
\addcontentsline{toc}{section}{References}
\bibliography{document}
\bibliographystyle{ieeetr}

%\newpage
%\addcontentsline{toc}{section}{Appendix A}
%\section*{Appendix A} \label{ch6}

\label{endOfDoc}
\end{document}
